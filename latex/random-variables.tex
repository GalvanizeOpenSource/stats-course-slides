\section{Random Variables}

\begin{frame}
A random variable $X$ is an object that can be used to generate numbers, in a way that valid probabilistic statements about the generated numbers can be made. 

\hfill

For example:
$$ P(X > 0) = 0.25 $$
$$ P(-1 < X < 1) = 0.25 $$
$$ P(X < -1) = 0.02 $$
$$ P(X > 1 \mid X > 0) = 0.5 $$

Are all probabilistic statements about an unknown random variable $X$.
\end{frame}
%

%
\begin{frame}
Random variables can be used to model real life events or measurements when
there is some variation in their outcomes that we cannot account for

\begin{itemize}
\item The number of heads seen in ten flips of a quarter.
\item The number of heads seen in ten flips of a dime.
\item The number of buses that arrive late to a stop in Seattle in a single
day.
\item The number of times my cat asks for food between 5 and 6 pm (when she is
always fed) in a given day.
\item The temperature on a mid-summer's day in Seattle.
\item The rainfall in a mid-winter's day in Seattle.
\end{itemize}

\end{frame}
%

%
\begin{frame}
We naturally have a feeling that there is something \textbf{the same} about
these two situations

\begin{itemize}
\item The number of heads seen in ten flips of a quarter.
\item The number of heads seen in ten flips of a dime.
\end{itemize}

What is it?
\end{frame}
%

%
\begin{frame}

We expect that the probabilities

$$ P(\text{We get 5 heads in 10 flips of a quarter}) $$
$$ P(\text{We get 5 heads in 10 flips of a dime}) $$

are \textbf{equal}.

\hfill

So are

$$ P(\text{We get 2 heads in 10 flips of a quarter}) $$
$$ P(\text{We get 2 heads in 10 flips of a dime}) $$

and so on...
\end{frame}
%

%
\begin{frame}
The \textbf{sameness} that we sense between

\begin{itemize}
\item The number of heads seen in ten flips of a quarter.
\item The number of heads seen in ten flips of a dime.
\end{itemize}

is that the probabilities of all events like

$$ P(\text{We get } N \text{ heads in 10 flips of a quarter}) $$
$$ P(\text{We get } N \text{ heads in 10 flips of a dime}) $$

\textbf{are all equal}.
\end{frame}
%

%
\begin{frame}
We summarize this by saying that the random variables


\begin{itemize}
\item The number of heads seen in ten flips of a quarter.
\item The number of heads seen in ten flips of a dime.
\end{itemize}

\textbf{have the same distribution}.

\hfill

The \textbf{distribution} of a random variable is the pattern of all
probabilities we assign to all outcomes of the random variable.
\end{frame}

%
\begin{frame}
So two random variables have the same distribution if \textbf{they assign the
same probabilities to all of the possible outcomes}.

\hfill

In this case, we use the shorthand \textbf{equally distributed}.
\end{frame}
%
